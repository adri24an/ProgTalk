\chapter{Especificación de Requisitos}
\label{ch:requisitos}
\lstsetpt

El objetivo de nuestro proyecto es la elaboración de una herramienta
(a la que a partir de ahora nos referiremos como \textit{Progtalk}) a
través de la cual seamos capaces de crear una representación gráfica
de una comunicación entre varias entidades con el objetivo de
facilitar la compresión de dicha comunicación. Esto es especialmente
útil en comunicaciones en las que haya multitud de entidades
participantes y/o un gran número de intercambio de mensajes.

Por ejemplo, una comunicación del tipo ``Individuo A saluda al
individuo B'' sería procesada por nuestra herramienta dando como
resultado una representación gráfica como ésta:

\todo{AB: Crear ejemplo con nuestro programa. Para ello que Angel me diga
  como lo hace el!}

Todas las comunicaciones que queramos procesar con la herramienta
\textit{Progtalk} deben de estar traducidas a un lenguaje concreto que
hemos creado para la ocasión llamado \textit{MSCLan}. A lo largo de
este capítulo iremos detallando las características de este lenguaje.

Por otro lado cómo ya comentamos en el capitulo anterior~\ref{ch:msc}
hemos escogido los \textit{diagramas de secuencia} (a partir de ahora
\textit{MSC, Message Sequence Charts}) para representar gráficamente
las comunicaciones que el programa \textit{Progtalk} reciba. Como ya
vimos, según Bjorner~\cite{bjorner}, un \textit{MSC} es una notación
gráfica para representar intercambios de mensajes entre
entidades. Dentro del paradigma de nuestro problema llamaremos a estas
entidades o individuos \textit{instancias}~\ref{sec:Instancias}, y a
los mensajes que intercambian les mantendremos el nombre de
\textit{mensajes}~\ref{sec:Mensajes}. 

Cómo ya hemos comentado anteriormente hemos creado un lenguaje llamado
\textit{MSCLan} al que tendremos que traducir cualquier comunicación
que queramos que \textit{Progtalk} procese.

Nuestro lenguaje consta de dos secciones:
\begin{itemize}
\item Declaración de \textit{instancias}, y
\item declaración de \textit{mensajes}.
\end{itemize}

Cada una de estas declaraciones se llama sentencia, y entre sentencias
debe aparecer un símbolo ``;'' que ejerce de separador.

\section{Instancias}\label{sec:Instancias}

Las instancias de nuestro lenguaje, que como ya hemos dicho,
representan a los participantes de la comunicación que queremos
procesar, se describen mediante tres
parámetros:

\begin{itemize}
\item Identificador de instancia. Este parámetro debe ser
  obligatoriamente introducido por el usuario. Ademas debe ser un
  identificador válido (aquí se entiende como identificador valido un
  conjunto de caracteres y números que comienza obligatoriamente por
  una letra).
\item Tipo de la instancia. Este parámetro es opcional, y esta pensado
  darle una funcionalidad real en un futuro desarrollo.
\item Parámetro \textit{name}. Este parámetro es opcional, y esta
  pensado darle una funcionalidad real en un futuro desarrollo.
\end{itemize}

El formato de la declaración de una instancia es el siguiente\footnote{Los corchetes $[$ y $]$ indican la opcionalidad del parámetro en el resto de descripciones de la sintaxis.}:
\begin{center}
  \begin{minipage}{0.40\linewidth}
\begin{lstlisting}[mathescape]
instance $iid$ of $tid$ { $name$ }
\end{lstlisting}
  \end{minipage}
\end{center}
donde:
\begin{itemize}
\item \lstinline{instance}, \lstinline{of}, ``\lstinline!{!'' y
    ``\lstinline!}!'' son palabras reservadas del lenguaje, y
\item $iid$ (identificador de instancia), $tid$ (tipo de la
  instancia), y $name$ (nombre del parámetro) deben ser introducidos
  por el usuario.
\end{itemize}

Algunos ejemplos de declaración de instancias son:

\begin{lstlisting}
      instance a { "A" };
      instance b of type { "B" };
      instance c of control;
      instance d;
\end{lstlisting}

El usuario puede almacenar tantas instancias como desee, pero estas
deben tener un identificador de instancia único.

\section{Mensajes}\label{sec:Mensajes}

Los mensajes de nuestro lenguaje son la información que intercambian
los participantes de la comunicación que estamos procesando. Llamamos
\textit{evento} al envío o recepción de un mensaje.

Los mensajes se describen mediante seis parámetros:

\begin{itemize}
\item Identificador de mensaje: este parámetro es opcional. En caso de
  ser introducido debe ser un identificador válido (aquí se entiende
  como identificador válido cualquier conjunto de letras, números y
  ``\_'' que comience por una letra,
\item Descripción de mensaje\todo{Cuidado, he cambiado esto a descripción}: este parámetro es opcional, y almacena el mensaje de
  texto propiamente dicho. Puede ser cualquier conjunto de caracteres
  salvo las comillas y que empiece y termine por comillas,
\item Instancia origen del mensaje: este parámetro debe ser
  obligatoriamente introducido por el usuario, y representa la
  instancia que envió el mensaje. La instancia a la que se haga
  referencia en este parámetro, debe de haber sido declarada
  previamente en esa misma comunicación,
\item Instancia destino del mensaje: este parámetro debe ser
  obligatoriamente introducido por el usuario, y representa la
  instancia que recibió el mensaje. La instancia a la que se haga
  referencia en este parámetro, debe de haber sido declarada
  previamente en esa misma comunicación,
\item Tiempo de envío: este parámetro es opcional. Representa el
  momento exacto en que el mensaje fue enviado, y
\item Tiempo de recepción: este parámetro es opcional. Representa el
  momento exacto en que el mensaje fue recibido.
\end{itemize}

El formato de la declaración de un mensaje es el siguiente
\begin{center}
  \begin{minipage}{0.75\linewidth}
\begin{lstlisting}[mathescape]
message $[mid]$ $[$${ $desc$ }$]$ from $orig$ $[$@ $tsend$$]$ to $dest$ $[$@ $trec$$]$
\end{lstlisting}
  \end{minipage}
\end{center}
donde,
\begin{itemize}
\item \lstinline{message}, \lstinline{from}, \lstinline{to},
  ''\lstinline{@}'', ``\lstinline!{!'' y ``\lstinline!}!'' son
  palabras reservadas del lenguaje, y
\todo{AB: Qué hago para el tema de los corchetes para identificar
  parametros opcionales? Lo uso siempre?}
\item $mid$ (Identificador de mensaje\todo{Ortografía: cuidado con el uso de mayúsculas}), $desc$ (descripción del mensaje), $orig$
  (Instancia origen del mensaje), $tsend$ (Tiempo de envío), $dest$
  (Instancia destino del mensaje), y $trec$ (Tiempo de recepción)
  deben ser introducidos por el usuario.
\end{itemize}

La instancia origen del mensaje y el instante de envío forman un
evento de envío, mientras que la instancia destino del mensaje y el
instante de recepción forman un evento de recepción.

Cabe destacar que es responsabilidad del usuario traducir la
comunicación que éste quiere pasar como entrada a \textit{Progtalk} a
\textit{MSCLan}.

Algunos ejemplos de declaración de mensajes son:\todo{Atento al espaciado en todos los demás ejemplos: coherencia.}

\begin{lstlisting}
message m1 { "request send to dhl" } from u to a;
message m2 { "send" } from a @ m1? to f @ +2;
message { "internal" } from a @ m2!+3 to a @ m1?+5;
\end{lstlisting}

\section{Especificaciones de tiempos}
\label{sec:Tiempos}

Esta sección se refiere a los instantes de tiempo en que se envían y
reciben los mensajes (sección~\ref{sec:Mensajes})\todo{Todo así.} que las
instancias~\ref{sec:Instancias} participantes en la
comunicación intercambian. Como ya comentamos anteriormente, nos
referiremos al envío o recepción de un mensaje como a un
\textit{evento}.

Como vimos en la sintaxis anterior:
\begin{center}
  \begin{minipage}{0.75\linewidth}
\begin{lstlisting}[mathescape]
message $[mid]$ $[${ $sms$ }$]$ from $orig$ $[$@ $tsend$$]$ to $dest$ $[$@ $trec$$]$
\end{lstlisting}
  \end{minipage}
\end{center}
la especificación del tiempo del evento de envío es $tsend$ y la del tiempo del evento de recepción es $trec$.
 Los tiempos en que suceden los eventos pueden ser
especificados explicitamente en la declaración de un mensaje o no, en
cuyo caso son calculados implicitamente por \textit{Progtalk}.

Los tiempos de un evento se pueden especificar de forma absoluta o relativa a otro evento.

\todo{¿Los tiempos se miden en enteros? Decirlo.}
\subsection{Tiempos Absolutos}

Entenderemos como tiempos absolutos los siguiente casos:

\begin{itemize}
\item Cuando en la comunicación aparece asociado a un evento
  específicamente un valor de tiempo, o
\item si estamos hablando del instante de envío del primer mensaje de
  la comunicación y en la declaración de este, no aparece ningún valor. En
  este caso especial consideraremos siempre un valor absoluto de cero.
\end{itemize}

Un ejemplo del uso de tiempos absolutos es:

\begin{lstlisting}
    message m1 { "hola" } from programaA @ 2 to programaB @ 5;
\end{lstlisting}

\noindent donde el mensaje \lstinline{m1} consiste en el envío del mensaje
\lstinline{"hola"} de la instancia \lstinline{programaA} en el
instante \textit{2} y la recepción del mensaje en la instancia
\lstinline{programaB} en el instante \textit{5}.

\subsection{Tiempos relativos}

Por otro lado entenderemos por tiempos relativos los siguiente casos:

\begin{itemize}
\item Relativo implícito: cuando un evento de la comunicación no tiene
  un valor especifico, en cuyo caso lo calcularemos sumándole una unidad al
  valor asociado al evento inmediatamente anterior (ya sea envío del
  mensaje actual, o recepción del mensaje anterior)\todo{Un ejemplo de cada.}. Por ejemplo:

  \begin{lstlisting}
    message m1 { "hola" } from programaA @ 2 to programaB;
  \end{lstlisting}

  Aquí el tiempo de recepción del mensaje se calcula a partir del
  tiempo de envío de ese mismo mensaje más una unidad, es decir, que
  el instante de recepción en \lstinline{programaB} es \textit{3}.

\item Relativo explicito simple: si en la declaración de un mensaje
  aparece como instante en que ocurre un evento un tiempo del tipo
  \lstinline[mathescape]{+$t$}\todo{Sigue este esquema en el resto de las descripciones de tiempos} donde $t$ es un numero 
  entero, pero no se especifica un evento de referencia. En este caso
  calcularemos el parámetro como el instante del ultimo evento (ya
  sea envío del mensaje actual, o recepción del mensaje anterior)
  mas el valor $t$ que aparece en la declaración. Por  
  ejemplo:
      
  \begin{lstlisting}
    message m1 { "hola" } from programaA @ 2 to programaB @ +3;
  \end{lstlisting}
      
  Aquí el tiempo de recepción del mensaje se calcula a partir del
  mensaje de envío más \textit{3}, es decir, que el instante de
  recepción en \lstinline{programaB} es \textit{5}.
      \todo{Un ``relativo implícito'' es como un ``relativo explícito simple'' con un +1}
\item Relativo explicito complejo: si en la declaración de un mensaje
  aparece como instante en que ocurre un evento otro evento de
  referencia y un valor del tipo \lstinline{+unidad} donde
  \lstinline{unidad} es un numero entero. El evento de referencia
  especificado por el usuario\todo{Te recuerdo que esto no te gusta} puede referirse a un evento de envío o
  de recepción\todo{Explicar cuál es la notación para referirse a eventos: \lstinline{!} y \lstinline{?}}, y debe de pertenecer a un mensaje ya declarado
  anteriormente en esa comunicación. En este caso calcularemos el
  parámetro tomando el instante en que ocurrió el evento que se
  especifica como referencia y sumándole el valor \lstinline{+unidad}
  especificado también en la declaración del mensaje. Cabe destacar
  que existe la posibilidad de que sólo se introduzca el evento de
  referencia y que no aparezca un valor del tipo \lstinline{{+unidad},
    en cuyo caso consideraremos dicho valor \lstinline{+unidad} como
    \textit{+0}. 
  Por ejemplo:

  \begin{lstlisting}
  message m1 { "hello" } from programaA @ 2 to programaB @ +5;
  message m2 { "goodbye" } from programaB @ m1!+3 to programaA @ m1?;
  \end{lstlisting}

  Aquí el tiempo de envío del mensaje \lstinline{m2} se calcula sumándole al tiempo de envío del
  mensaje \lstinline{m1} (es decir, \textit{2}) el valor \textit{+3}
  dando como resultado \textit{5}. El tiempo de recepción se calculará
  sumándole al tiempo de recepción de \lstinline{m1} (es decir,
  \textit{7})), el valor \textit{+0} dando como resultado \textit{7}.
\end{itemize}

\todo{AB: hacer imagenes ilustrativas para cada caso}

\section{Sintaxis Concreta}

La gramática, en formato EBNF (\emph{Extended Backus-Naur Form}) de
\textit{MSCLan} es la siguiente:
\begin{lstlisting}[style=spec, language={}]
    msc ::= inst_decl* message*
    inst_decl ::= INSTANCE iid |
                  INSTANCE iid OF tid; |
                  INSTANCE iid {string}; |
                  INSTANCE iid OF tid {string};               
    message ::= MESSAGE mid_opt string_opt origin destiny;
    mid_opt ::= LAMBDA | mid
    string_opt ::= LAMBDA | {string}
    origin ::= LAMBDA | origin_opt
    origin_opt ::= FROM iid time_ref_opt
    destiny ::= LAMBDA | destiny_opt
    destiny_opt ::= TO iid time_ref_opt
    time_ref_opt ::= LAMBDA | @ time_ref
    time_ref ::= abs_time | rel_time
    abs_time ::= num
    rel_time ::= ref dif_time_opt | dif_time
    ref ::= mid ! | mid ?
    dif_time_opt ::= LAMBDA | dif_time
    dif_time ::= + num | - num
    iid ::= ID
    tid ::= ID
    mid ::= ID
    string ::= STRING
    num ::= NUM
\end{lstlisting}

En ella hemos escrito con minúsculas los símbolos no terminales, y en 
mayúsculas los símbolos terminales. Cabe destacar que LAMBDA
representa la \textit{palabra vacía}.

Como vemos, la regla \textit{msc} esta formada de por un conjunto de
declaraciones de instancias seguido de un conjunto de declaraciones de
mensajes. Los distintos formatos de intancias y mensajes ya han sido
explicados anteriormente en este capítulo.

%%% Local Variables: 
%%% mode: latex
%%% TeX-master: "progtalk"
%%% TeX-PDF-mode: t
%%% ispell-local-dictionary: "castellano"
%%% End: 
