\chapter*{Resumen}
\addcontentsline{toc}{chapter}{Resumen}

El formalismo de los diagramas de secuencia (\emph{Message Sequence Charts})
es un lenguaje gráfico que permite especificar escenarios de
interacción entre componentes de un sistema\todo{Cita a Harel, ver el
  libro de Björner}. Es un formalismo extraordinariamente popular
aunque también es usado en innumerables versiones con innumerables
interpretaciones.

En este trabajo fin de carrera hemos
\begin{enumerate}
\item definido una \textbf{notación textual concreta} para especificar
 diagramas de secuencia y
\item hemos implementado un programa que admite como entrada un
  diagrama de secuencia escrito en dicha notación y que \textbf{genera
  visualizaciones gráficas} del mismo en distintos formatos gráficos.
\end{enumerate}

El uso para el que el programa está diseñado es para mantener
descripciones de diagramas de secuencia fáciles de modificar por un
ser humano y que resulte sencillo de seguir sus cambios (con un
sistema de control de versiones, por ejemplo) y para poder mostrar
dichos diagramas visualmente.

Además, el programa realiza un análisis sobre la validez de dichas
descripciones como por ejemplo impidiendo describir mensajes que
viajen hacia atrás en el tiempo o con origen o destino no definido.

\todo{AH: A partir de aquí conservo lo que hizo Adrian pero creo que
  podemos eliminarlo}

Normalmente la comunicación entre programas, procesos, etc, es larga y
compleja lo que hace tedioso para un usuario su análisis e
interpretación. Debido a esto nos hemos propuesto el diseño e
implementación de una herramienta que permita al usuario representar
graficamente una comunicación, facilitándole así la interpretación de
ésta. 

El esquema gráfico que hemos elegido como salida de nuestra
herramienta es el de los diagramas de secuencia (\textit{MSC}, Message
Sequence Charts) debido a que nos pareció una opción muy sencilla e
intuitiva a la hora de representar intercambios de mensajes entre
entidades. Los MSC son diagramas usados para modelar la interacción
entre diferentes objetos dentro de un sistema basado en UML. En el
capítulo \ref{ch:msc} explicaremos más en detalle este tema.

Durante la creación de nuestra herramienta hemos creado un lenguaje 
al cual deberá ser traducida toda comunicación que queramos procesar y
representar gráficamente.

Nuestra herramienta, que hemos bautizado con el nombre de
\textit{Progtalk} se integra dentro del proyecto \textit{Transformers}. 
\todo{AB:Aquí deberíamos explicar en detalle que papel nuestro 
juega Progtalk en Transformers.}

El funcionamiento de Progtalk se compone de tres fases:

\begin{itemize}
\item Validación léxica y sintáctica del fichero de entrada, el cuál
  contiene la comunicación,
\item validación semántica, almacenaje de la información parseada, y
  comprobación de la integridad de la información almacenada, y
\item exportación de la comunicación a un fichero externo de tipo
  \textit{latex}.
\end{itemize}

El presente trabajo esta estructurado del siguiente modo. En primer
lugar, en el capítulo \ref{ch:msc} se presentan los \textit{Diagramas
  de Secuencia}, su historia y un resumen de su funcionamiento. Tras
esta introducción teórica pasaremos al capitulo \ref{ch:requisitos},
donde explicaremos en detalle la especificación de requisitos del
proyecto. En el capitulo \ref{ch:diseno} explicaremos el
análisis y diseño que se ha realizado antes de comenzar la
implementación de la herramienta. Por último, en el capitulo
\ref{ch:conclusiones} se presentan las conclusiones extraídas de la
realización de este proyecto, y posibles vías futuras de desarrollo.

%%% Local Variables: 
%%% mode: latex
%%% TeX-master: "progtalk"
%%% TeX-PDF-mode: t
%%% ispell-local-dictionary: "castellano"
%%% End: 
