\chapter*{Resumen}
\addcontentsline{toc}{chapter}{Resumen}

Este proyecto
consiste en el diseño e implementación de una herramienta la cual a
partir de una comunicación entre varias entidades (programas,
procesos, etc), nos devuelve una representación gráfica de dicha
comunicación. \textit{Progtalk} se integra dentro del proyecto
\textit{Transformers}, en el cual se estudian varios lenguajes con el
objetivo de representar la información de los requisitos del sistema
de ferrocarril europeo (ERTMS). En concreto, uno de los lenguajes que
se estudia es el de diagramas de secuencia (MSC, \emph{Message
  Sequence Charts}), y aquí es donde encaja
\textit{Progtalk}. \todo{AB:Aquí deberíamos explicar en detalle que
  papel juega Progtalk en Transformers.}

La labor de nuestra herramienta consiste en, dada una comunicación
entre dos o mas entidades, el procesado de esta comunicación y la
creación de una representación gráfica de ésta. Para ello podemos
describir el funcionamiento de la herramienta en tres fases:

\begin{itemize}
\item Validación léxica y sintáctica del fichero de entrada, el cuál
  contiene la comunicación,
\item validación semántica, almacenaje de la información parseada, y
  comprobación de la integridad de la información almacenada, y
\item exportación de la comunicación a un fichero externo de tipo
  \textit{latex}.
\end{itemize}

El presente trabajo esta estructurado del siguiente modo. En primer
lugar, en el capítulo \ref{ch:msc} se presentan los \textit{Diagramas
  de Secuencia}, su historia y un resumen de su funcionamiento. Tras
esta introducción teórica pasaremos al capitulo \ref{ch:lenguaje},
donde explicaremos en detalle en qué consiste el lenguaje que hemos
creado para representar las comunicaciones entrantes a
\textit{Progtalk}. En el capitulo \ref{ch:diseno} explicaremos el
análisis y diseño que se ha realizado antes de comenzar la
implementación de la herramienta. Por último, en el capitulo
\ref{ch:conclusiones} se presentan las conclusiones extraídas de la
realización de este proyecto, y posibles vías futuras de desarrollo.

%%% Local Variables: 
%%% mode: latex
%%% TeX-master: "progtalk"
%%% TeX-PDF-mode: t
%%% ispell-local-dictionary: "castellano"
%%% End: 
