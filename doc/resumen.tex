\chapter*{Resumen}
\addcontentsline{toc}{chapter}{Resumen}

Normalmente la comunicación entre programas, procesos, etc, es larga y
compleja lo que hace tedioso para un usuario su análisis e
interpretación. Debido a esto nos hemos propuesto el diseño e
implementación de una herramienta que permita al usuario representar
graficamente una comunicación, facilitándole así la interpretación de
ésta. 

El esquema gráfico que hemos elegido como salida de nuestra
herramienta es el de los diagramas de secuencia (MSC, Message Sequence
Charts) debido a que nos pareció una opción muy sencilla e intuitiva
a la hora de representar intercambios de mensajes entre entidades.

Hemos creado un lenguaje durante la creación de nuestra herramienta
al cual deberá ser traducida toda comunicación que queramos procesar.

Nuestra herramienta, que hemos bautizado con el nombre de
\textit{Progtalk} se integra dentro del proyecto \textit{Transformers}. 
\todo{AB:Aquí deberíamos explicar en detalle que papel nuestro 
juega Progtalk en Transformers.}

El funcionamiento de Progtalk se compone de tres fases:

\begin{itemize}
\item Validación léxica y sintáctica del fichero de entrada, el cuál
  contiene la comunicación,
\item validación semántica, almacenaje de la información parseada, y
  comprobación de la integridad de la información almacenada, y
\item exportación de la comunicación a un fichero externo de tipo
  \textit{latex}.
\end{itemize}

El presente trabajo esta estructurado del siguiente modo. En primer
lugar, en el capítulo \ref{ch:msc} se presentan los \textit{Diagramas
  de Secuencia}, su historia y un resumen de su funcionamiento. Tras
esta introducción teórica pasaremos al capitulo \ref{ch:lenguaje},
donde explicaremos en detalle en qué consiste el lenguaje que hemos
creado para representar las comunicaciones entrantes a
\textit{Progtalk}. En el capitulo \ref{ch:diseno} explicaremos el
análisis y diseño que se ha realizado antes de comenzar la
implementación de la herramienta. Por último, en el capitulo
\ref{ch:conclusiones} se presentan las conclusiones extraídas de la
realización de este proyecto, y posibles vías futuras de desarrollo.

%%% Local Variables: 
%%% mode: latex
%%% TeX-master: "progtalk"
%%% TeX-PDF-mode: t
%%% ispell-local-dictionary: "castellano"
%%% End: 
