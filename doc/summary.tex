\begin{otherlanguage}{english}

\chapter*{Summary}
\addcontentsline{toc}{chapter}{Summary}

\emph{Message Sequence Charts} formalism\todo{AB: comprobar traduccion} is
a graphic language that allows the specification of interaction
scenarios between system components\todo{Cita a Harel, ver el libro de
Björner}. This formalism is extremely popular but can be used in
innumerable versions and with innumerable interpretations too.

In this dissertation\todo{AB: comprobar traduccion} we have
\begin{enumerate}
\item defined a \textbf{concrete textual notation} to specify message
  sequence charts and
\item implemented a program that admits as input a message sequence
  chart written in this notation and \textbf{returns a graphic
  visualization} of it on different formats.
 
The purpose of this program is to allow the maintenance of easy
adaptable message sequence charts descriptions by a human facilitating
its modifications monitorization (e.g. with a versions control system)
and allowing the generation of graphic representations of this
diagrams.

In addition, the program makes an analysis of the validity of this
representations not allowing for example descriptions of messages been
sent backwards in time or with an origin/destiny not specified.

\todo{AB: traducir lo demas?}

\end{otherlanguage}

%%% Local Variables: 
%%% mode: latex
%%% TeX-master: "progtalk"
%%% TeX-PDF-mode: t
%%% ispell-local-dictionary: "castellano"
%%% End: 
