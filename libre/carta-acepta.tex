\documentclass[12pt]{article}

\usepackage{a4wide}
\usepackage[latin1]{inputenc}

% \usepackage{graphics} % Or
% \usepackage{graphicx} %% and then \includegraphics[scale=0.5,...]{f.eps}

\usepackage[spanish]{babel}
%\usepackage[spanish,english]{babel}

% \usepackage{listings}

% \usepackage{url}
% \usepackage{html}

% \usepackage{fancybox}
% \usepackage{fancyheadings}

% \usepackage{courier}
% \usepackage{times}

\begin{document}

\thispagestyle{empty}
\pagestyle{empty}

\mbox{}
\vspace{5cm}

\begin{center}
  \LARGE
  \underline{Colaboraci�n en Proyecto}
\end{center}

\begin{center}
  \large
  14 de julio de 2010
\end{center}


El alumno \emph{Adrin Bartol Molina} con D.N.I. n�mero
\emph{51459060X} est� colaborando en el proyecto \emph{Transformers}
dirigido por �ngel Herranz y desarrollado en el Departamento de
Lenguajes y Sistemas Inform�ticos e Ingenier�a de Software, dentro del
grupo de investigaci�n Babel desarrollando las siguientes tareas:

%emph{Generalised Object-oriented Normal Form} (GONF).
\section*{Tareas}
\begin{itemize}
\item Dise�o de un lenguaje concreto para la representaci�n de MSCs (\emph{Message Sequence Charts}),
\item an�lisis sint�ctico de dicho lenguaje, y
\item traducci�n a distintos formatos para una visualizaci�n gr�fica
  de los diagramas (LaTeX, PDF, PNG, Dot, etc.).
\end{itemize}

\vspace{2cm}
Fdo. D. �ngel Herranz Nieva

\end{document}

%%% Local Variables: 
%%% mode: latex
%%% TeX-master: t
%%% TeX-PDF-mode: t
%%% End: 
